\documentclass[a4j]{jarticle}

\usepackage{kut-abstract}
\usepackage[dvipdfmx]{graphicx}
\usepackage{lmodern}
\usepackage{textcomp}
\usepackage{latexsym}
\usepackage{url}

\ScInfo
\Bachelor	%% 卒業研究論文梗概の場合
%\Project	%% プロジェクト研究報告書梗概の場合
%\Seminar	%% 特別研究セミナー課題研究報告書梗概の場合
%\Master	%% 修士学位論文(情報システム工学コース)梗概の場合
%\Doctorate	%% 博士学位論文(情報システム工学コース)梗概の場合
%\English	%% 英語の場合

\years{平成xx}
\Eyears{xxxx}
\title{Title}
\Etitle{Title}
\idnumber{Your ID}
\author{Your Name}
\Eauthor{Your Name}
\affiliate{Laboratory name}
\Eaffiliate{Laboratory name}

\begin{document}
\begin{Abstract}

  \section{はじめに}

  本論文は,高知工科大学情報学群 \cite{kut} の論文テンプレートである.
  \TeX ファイルの内容は,内容に応じて自由に変更して構わない.
  ただし,\TeX ファイルからの参考文献の参照が一つもないと,
  参考文献リストを BiBTeX で出力している関係から,うまくコンパイルできないので注意すること.
  したがって,他の参考文献を追加した後 \verb|\cite{kut}| を消せば良い.

  \begin{figure}[ht]
    \centering
    \includegraphics[width=8cm]{images/Parrots.bmp}
    \caption{サンプル画像}
    \label{fig:sample}
  \end{figure}

  % 参考文献リスト
  \bibliography{refs}
  \bibliographystyle{sieicej}

  % 参考文献の文字間隔が広く空いて出力される場合は,以下を使う
  %
  % \begin{flushleft}
  %   \bibliography{refs}
  %   \bibliographystyle{sieicej}
  % \end{flushleft}

\end{Abstract}
\end{document}
